% Informationen ------------------------------------------------------------
% 	Definition von globalen Parametern, die im gesamten Dokument verwendet
% 	werden können (z.B auf dem Deckblatt etc.).
% --------------------------------------------------------------------------
\newcommand{\titel}{Dokumentation}
\newcommand{\untertitel}{Praktikum Robotik}
\newcommand{\art}{}
\newcommand{\fachgebiet}{Informatik}
\newcommand{\autor}{Bj\"orn Borgmann, Norman Caspari, Matthias Larisch, Philip Weber}
\newcommand{\studienbereich}{Informatik}
\newcommand{\matrikelnr}{}
\newcommand{\erstgutachter}{Prof. Dr.-Ing. Andreas Hein}
\newcommand{\zweitgutachter}{Dipl.-Inform. Melvin Isken}
\newcommand{\jahr}{2012}

% Eigene Befehle
\newcommand{\todo}[1]{\textbf{\textsc{\textcolor{red}{(TODO: #1)}}}}

% Autorennamen in small caps
\newcommand{\AutorZ}[1]{\textsc{#1}}
\newcommand{\Autor}[1]{\AutorZ{\citeauthor{#1}}}

% Befehle zur semantischen Auszeichnung von Text
\newcommand{\NeuerBegriff}[1]{\textbf{#1}}
\newcommand{\Fachbegriff}[1]{\textit{#1}}
\newcommand{\Prozess}[1]{\textit{#1}}
\newcommand{\Webservice}[1]{\textit{#1}}
\newcommand{\Eingabe}[1]{\texttt{#1}}
\newcommand{\Code}[1]{\texttt{#1}}
\newcommand{\Datei}[1]{\texttt{#1}}
\newcommand{\Datentyp}[1]{\textsf{#1}}
\newcommand{\XMLElement}[1]{\textsf{#1}}

% Abkürzungen
\newcommand{\vgl}{Vgl.\ }
\newcommand{\ua}{\mbox{u.\,a.\ }}
\newcommand{\zB}{\mbox{z.\,B.\ }}
\newcommand{\bs}{$\backslash$}



% Dokumentenkopf -----------------------------------------------------------
% 	Diese Vorlage basiert auf "scrreprt" aus dem koma-script.
%		Die Option draft sollte beim fertigen Dokument ausgeschaltet werden.
% --------------------------------------------------------------------------
\documentclass[
	12pt,						% Schriftgr��e
	DIV10,
	german,					% f�r Umlaute, Silbentrennung etc.
	a4paper,					% Papierformat
	oneside,					% einseitiges Dokument
	titlepage,				% es wird eine Titelseite verwendet
	halfparskip,			% Abstand zwischen Abs�tzen (halbe Zeile)
	normalheadings,		% Gr��e der �berschriften verkleinern
	liststotoc,				% Verzeichnisse im Inhaltsverzeichnis auff�hren
	bibtotoc,				% Literaturverzeichnis im Inhaltsverzeichnis auff�hren
	idxtotoc,				% Index im Inhaltsverzeichnis auff�hren
	tablecaptionabove,	% Beschriftung von Tabellen oberhalb ausgeben
	final						% Status des Dokuments (final/draft)
]{scrreprt}
% Anpassung des Seitenlayouts ----------------------------------------------
% 	siehe Seitenstil.tex
% --------------------------------------------------------------------------
\usepackage[
	automark,			% Kapitelangaben in Kopfzeile automatisch erstellen
	headsepline,	% Trennlinie unter Kopfzeile
	ilines				% Trennlinie linksb�ndig ausrichten
]{scrpage2}


% Anpassung an Landessprache -----------------------------------------------
% 	Verwendet globale Option german siehe \documentclass
% --------------------------------------------------------------------------
\usepackage{babel}


% Umlaute ------------------------------------------------------------------
% 		Umlaute/Sonderzeichen wie ���� direkt im Quelltext verwenden (CodePage).
%		Erlaubt automatische Trennung von Worten mit Umlauten.
% --------------------------------------------------------------------------
\usepackage[utf8]{inputenc}
\usepackage[T1]{fontenc}
%\usepackage{ae} % "sch�neres" �
%\usepackage{textcomp} % Euro-Zeichen etc.

% Grafiken -----------------------------------------------------------------
% 		Einbinden von Grafiken [draft oder final]
% 		Option [draft] bindet Bilder nicht ein - auch globale Option
% --------------------------------------------------------------------------
%\usepackage[dvips,final]{graphicx}
\usepackage{graphicx}
\graphicspath{{Bilder/}} % Dort liegen die Bilder des Dokuments

% Zum einbinden von PDFs
\usepackage{pdfpages}

% Befehle aus AMSTeX f�r mathematische Symbole z.B. \boldsymbol \mathbb ----
\usepackage{amsmath,amsfonts}

% F�r Index-Ausgabe; \printindex -------------------------------------------
\usepackage{makeidx}

% Einfache Definition der Zeilenabst�nde und Seitenr�nder etc. -------------
\usepackage{setspace}
\usepackage{geometry}


% Symbolverzeichnis --------------------------------------------------------
% 	Symbolverzeichnisse bequem erstellen, beruht auf MakeIndex.
% 		makeindex.exe %Name%.nlo -s nomencl.ist -o %Name%.nls
% 	erzeugt dann das Verzeichnis. Dieser Befehl kann z.B. im TeXnicCenter
%		als Postprozessor eingetragen werden, damit er nicht st�ndig manuell
%		ausgef�hrt werden muss.
%		Die Definitionen sind ausgegliedert in die Datei Abkuerzungen.tex.
% --------------------------------------------------------------------------
\usepackage{nomencl}
  \let\abbrev\nomenclature
  \renewcommand{\nomname}{Abk�rzungsverzeichnis}
  \setlength{\nomlabelwidth}{.25\hsize}
  \renewcommand{\nomlabel}[1]{#1 \dotfill}
  \setlength{\nomitemsep}{-\parsep}


% Zum Umflie�en von Bildern -------------------------------------------------
%\usepackage{floatflt}


% Zum Einbinden von Programmcode --------------------------------------------
\usepackage{listings}
\usepackage{xcolor} 
\definecolor{hellgelb}{rgb}{1,1,0.9}
\definecolor{colKeys}{rgb}{0,0,1}
\definecolor{colIdentifier}{rgb}{0,0,0}
\definecolor{colComments}{rgb}{1,0,0}
\definecolor{colString}{rgb}{0,0.5,0}
\lstset{%
    float=hbp,%
    basicstyle=\texttt\small, %
    identifierstyle=\color{colIdentifier}, %
    keywordstyle=\color{colKeys}, %
    stringstyle=\color{colString}, %
    commentstyle=\color{colComments}, %
    columns=flexible, %
    tabsize=2, %
    frame=single, %
    extendedchars=true, %
    showspaces=false, %
    showstringspaces=false, %
    numbers=left, %
    numberstyle=\tiny, %
    breaklines=true, %
    backgroundcolor=\color{hellgelb}, %
    breakautoindent=true, %
%    captionpos=b%
}

% Lange URLs umbrechen etc. -------------------------------------------------
\usepackage{url}


% Wichtig f�r korrekte Zitierweise ------------------------------------------
\usepackage[square]{natbib}
% Quellenangaben in eckige Klammern fassen ----------------------------------
\bibpunct{[}{]}{;}{a}{}{,~}


% PDF-Optionen --------------------------------------------------------------
\usepackage[
bookmarks,
bookmarksopen=true,
pdftitle={\titel},
pdfauthor={\autor},
pdfcreator={\autor},
pdfsubject={\titel},
pdfkeywords={\titel},
colorlinks=true,
linkcolor=black, % einfache interne Verkn�pfungen
anchorcolor=black,% Ankertext
citecolor=blue, % Verweise auf Literaturverzeichniseintr�ge im Text
filecolor=magenta, % Verkn�pfungen, die lokale Dateien �ffnen
menucolor=black, % Acrobat-Men�punkte
urlcolor=cyan, 
% f�r die Druckversion k�nnen die Farben ausgeschaltet werden:
%linkcolor=black, % einfache interne Verkn�pfungen
%anchorcolor=black,% Ankertext
%citecolor=black, % Verweise auf Literaturverzeichniseintr�ge im Text
%filecolor=black, % Verkn�pfungen, die lokale Dateien �ffnen
%menucolor=black, % Acrobat-Men�punkte
%urlcolor=black, 
backref,
%pagebackref,
plainpages=false,% zur korrekten Erstellung der Bookmarks
pdfpagelabels,% zur korrekten Erstellung der Bookmarks
hypertexnames=false,% zur korrekten Erstellung der Bookmarks
linktocpage % Seitenzahlen anstatt Text im Inhaltsverzeichnis verlinken
]{hyperref}

% Zum fortlaufenden Durchnummerieren der Fu�noten ---------------------------
%\usepackage{chngcntr}


% f�r lange Tabellen
\usepackage{longtable}
\usepackage{array}
\usepackage{ragged2e}
\usepackage{lscape}
% f�r Tabellen mit variabler Spaltenbreite
%\usepackage{tabularx}

% Spaltendefinition rechtsb�ndig mit definierter Breite ---------------------
\newcolumntype{w}[1]{>{\raggedleft\hspace{0pt}}p{#1}}

% Formatierung von Listen �ndern
\usepackage{paralist}
% Standardeinstellungen:
% \setdefaultleftmargin{2.5em}{2.2em}{1.87em}{1.7em}{1em}{1em}

%bringt das H Kommando zur Bild Plazierung mit
%\usepackage{here} 

%\usepackage[utf8]{inputenc}

% Erstellung eines Index und Abk�rzungsverzeichnisses aktivieren -----------
\makeindex
%\makenomenclature


% Kopf- und Fu�zeilen, Seitenr�nder etc. -----------------------------------
% Zeilenabstand ------------------------------------------------------------
\onehalfspacing

% Seitenr�nder -------------------------------------------------------------
\geometry{paper=a4paper,left=35mm,right=35mm,top=30mm}

% Kopf- und Fu�zeilen ------------------------------------------------------
\pagestyle{scrheadings}

% Kopf- und Fu�zeile auch auf Kapitelanfangsseiten -------------------------
\renewcommand*{\chapterpagestyle}{scrheadings}

% Schriftform der Kopfzeile ------------------------------------------------
\renewcommand{\headfont}{\normalfont}

% Kopfzeile ----------------------------------------------------------------
\ihead{\large{\textsc{Seminararbeit}}\\	\small{Hierachisches Clustering interstellarer Objekte} \\[2ex] \textit{\headmark}}
\chead{}
\ohead{\includegraphics[scale=0.4]{UniOlLogoKlein.png}}
\setlength{\headheight}{21mm} % H�he der Kopfzeile
\setheadwidth[0pt]{textwithmarginpar} % Kopfzeile �ber den Text hinaus verbreitern
\setheadsepline[text]{0.4pt} % Trennlinie unter Kopfzeile

% Fu�zeile -----------------------------------------------------------------
\ifoot{}
\cfoot{}
\ofoot{\pagemark}

% erzeugt ein wenig mehr Platz hinter einem Punkt --------------------------
\frenchspacing 

% Schusterjungen und Hurenkinder vermeiden
\clubpenalty = 10000
\widowpenalty = 10000 
\displaywidowpenalty = 10000

% Quellcode-Ausgabe formatieren --------------------------------------------
\lstset{numbers=left, numberstyle=\tiny, numbersep=5pt, breaklines=true}
\lstset{emph={square}, emphstyle=\color{red}, emph={[2]root,base}, emphstyle={[2]\color{blue}}}

% Fu�noten fortlaufend durchnummerieren ------------------------------------
%\counterwithout{footnote}{chapter}

% kein page break bei chapter

\makeatletter
\renewcommand\chapter{\par%
    \global\@topnum\z@
    \@afterindentfalse
    \secdef\@chapter\@schapter}
\makeatother



% Eigene Definitionen f�r Silbentrennung
\hyphenation{Trenn-bar-es}
\hyphenation{J\"orgen}


% Das eigentliche Dokument -------------------------------------------------
%		Der eigentliche Inhalt des Dokuments beginnt hier. Die einzelnen Seiten
%		und Kapitel werden in eigene Dateien ausgelagert und hier nur inkludiert.
% --------------------------------------------------------------------------
\begin{document}

% auch subsubsection nummerieren
\setcounter{secnumdepth}{3}
\setcounter{tocdepth}{3}

% keine Kopf-/Fu�zeilen bei Deckblatt und Abstract
\ofoot{}
\thispagestyle{plain}
\begin{titlepage}

\begin{center}

\huge{\textsc{\textbf{\titel}}}\\[1.5ex]
\LARGE{\textbf{\untertitel}}\\[4ex]
\LARGE{\textbf{\art}}\\[6ex]%[1ex]
%\Large{im Fachgebiet \fachgebiet}\\[5ex]

\includegraphics[scale=0.25]{UniOlLogo.png}\\[2ex]

\normalsize
\begin{longtable}{rp{2.0ex}p{0.8\linewidth}}\\
%\begin{tabularx}{0.9\linewidth}{rp{2.0ex}X}\\
 %vorgelegt von:	& & \autor\\[1.2ex]
 Erstellt von:	& & \autor\\[1.2ex] %NEU
 Seminar: & & Data Mining in Astronomy\\[1.2ex]
 Studienbereich: & &\studienbereich\\[1.2ex]%[1.2ex]
 %Matrikelnummer: && \matrikelnr \\[1.2ex]
 %Erstgutachter:    &     & \erstgutachter \\[1.2ex]
 Professor:         && \erstgutachter\\[1.2ex]%[1.2ex] %NEU
 Tutor:     &    & \zweitgutachter \\[1.1ex]
%\end{tabularx}
\end{longtable}

%\copyright\ \jahr\\[1.5ex]

\end{center}

\singlespacing
\small
\noindent Dieses Werk einschließlich seiner Teile ist \textbf{urheberrechtlich geschützt}. Jede Verwertung außerhalb der engen Grenzen des Urheberrechtgesetzes ist ohne Zustimmung des Autors unzulässig und strafbar. Das gilt insbesondere für Vervielfältigungen, Übersetzungen, Mikroverfilmungen sowie die Einspeicherung und Verarbeitung in elektronischen Systemen.

\end{titlepage}

\ofoot{\pagemark}

% Seitennummerierung -------------------------------------------------------
%		Vor dem Hauptteil werden die Seiten in gro�en r�mischen Ziffern 
%		nummeriert...
% --------------------------------------------------------------------------
\pagenumbering{Roman}
\tableofcontents			% Inhaltsverzeichnis

% ...danach in normalen arabischen Ziffern ---------------------------------
\clearpage
\pagenumbering{arabic}

% Inhalt -------------------------------------------------------------------
%		Hier k�nnen jetzt die einzelnen Kapitel inkludiert werden. Sie m�ssen
%		in den entsprechenden .TEX-Dateien vorliegen. Die Dateinamen k�nnen
% 		nat�rlich angepasst werden.
% --------------------------------------------------------------------------
\chapter{Einleitung}

Diese Dokumentation beschreibt die Arbeitsergebnisse des Robotik Praktikums welches an der Universität Oldenburg im Rahmen des Sommersemesters 2012 durchgeführt wurde. Während des Praktikums wurde ein Turtlebot (http://www.willowgarage.com/turtlebot) verwendet und das darauf laufende ROS System (http://www.ros.org) entsprechend der Aufgabenstellung des Praktikums angepasst.

\section{Aufgabenstellung}

Die Schwerpunkte des Praktikums lagen bei der autonomen Erstellung von Karten und dem Navigieren in so zuvor kartierten Räumen. Dabei sollten neben der bereits auf dem Turtlebot vorhandene Sensorik, welche vor allem Odometrie Sensoren und eine Kinect umfasst, auch ein Laserscanner zur Anwendung kommen. Dieser war dafür in das vorhandene System zu integrieren. Wobei auch über eine Rauschreduzierung des Laserscanners nachgedacht werden sollte. Zur Steuerung der entwickelten Komponenten war es das Ziel eine grafische Oberfläche zu implementieren. Eine weitere Aufgabe des Praktikums war das erstellen von Videos welche die Arbeitsergebnisse dokumentieren.

% Literaturverzeichnis -----------------------------------------------------
%		Das Literaturverzeichnis wird aus der Datenbank Bibliographie.bib 
% 		erstellt. Die genaue Verwendung von bibtex wird hier jedoch nicht erkl�rt.
%		Link: http://de.wikipedia.org/wiki/BibTeX
% --------------------------------------------------------------------------
\bibliography{Bibliographie}
\bibliographystyle{natdin}		% DIN-Stil des Literaturverzeichnisses

%\include{Inhalt/Erklaerung}	% Selbst�ndigkeitserkl�rung (wird hier nicht gebraucht)

% Anhang -------------------------------------------------------------------
%		Die Inhalte des Anhangs werden analog zu den Kapiteln inkludiert.
%		Dies geschieht in der Datei Anhang.tex
% --------------------------------------------------------------------------
\begin{appendix}
	\clearpage
	\pagenumbering{roman}
	%\input{Inhalt/Anhang} %% F�r Anhang Kommentar hier entfernen
\end{appendix}


% Index --------------------------------------------------------------------
%		Zum Erstellen eines Index, die folgende Zeile auskommentieren.
% --------------------------------------------------------------------------
%\printindex		% Index hier einf�gen

\end{document}
