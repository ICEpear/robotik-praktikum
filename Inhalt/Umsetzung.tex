\section{Umsetzung}
Die Aufgabenstellung gliedert das Projekt in mehrere Teilaufgaben, welche getrennt voneinander umgesetzt werden.

\subsection{Laserscanner}
Der Laserscanner LZR-U901 benötigt einen Treiber, welcher von der Universität Bremen erhältlich ist.

Dieser Treiber liest alle vier Scan-Ebenen des Laserscanners ein und gibt diese sowohl als PointCloud2 als auch die erste Ebene als LaserScan in ROS aus.

Um bei der Navigation Hindernissen besser ausweichen zu können, ist es sinnvoll, auf mehr als eine Ebene zurückzugreifen. Hierzu wird der Treiber so erweitert, dass zusätzlich ein zweites Topic \emph{merged_scan} veröffentlicht wird, welches eine nach bestimmten Regeln erstellte Ebene darstellt.

Da das Ziel der Zusammenlegung der Ebenen auf Hinderniserkennung festgelegt ist, wird immer der nächstliegende Punkt aus allen vier Ebenen in die zusammengelegte Ebene eingefügt.

Die Datenrate des Laserscanners beträgt für alle 4 Ebenen 3,75 Bilder pro Sekunde bei einer Winkelauflösung von 0.3516°. Die Ebenen liegen im Winkel von 2° zueinander.



\subsection{Kinect}
Während der Entwicklung stellten sich schnell Probleme mit der Rechenleistung des Turtlebot-Netbooks heraus. Als einfache Gegenmaßnahme wird die Kinect mit verringerter Auflösung- und Framerate betrieben. Das Tiefenbild wird mit 160x120 Pixeln und das RGB Bild mit 640x480 Pixeln bei jeweils 15 Bildern pro Sekunde verarbeitet.

\subsection{Navigation}
Zur Navigation wird die ROS node move_base verwendet. Die Einstellungen aus dem mitgelieferten turtlebot_navigation werden übernommen. Änderungen dienen hauptsächlich dazu, den Laserscanner sowie die Kinect als Sensoren zur Hinderniserkennung einzusetzen. Zudem werden sensorspezifische Konfigurationsdaten wie z.B. Reichweite entsprechend gesetzt.

Damit die Kartenerstellung besser funktioniert, werden zudem einige Bewegungsparameter so verändert, dass langsamere und weniger ruckartige Bewegungen erfolgen.

\subsection{Kartenerstellung}
Die Kartenerstellung erfolgt durch gmapping, welches nach dem SLAM-Prinzip (simultaneous localization and mapping) arbeitet. Die Parameter werden mit kleinen Anpassungen von turtlebot_navigation übernommen.

\subsection{automatische Kartenerstellung}
Damit der Turtlebot eigenständig eine Karte aufbaut, wird das Paket exploration von Bosch verwendet. Dies arbeitet nach der frontier based exploration, welches ständig versucht, Punkte anzufahren, welche zwischen einem unbekannten und bekanntem Bereich liegen, jedoch keine Hindernisse sind.

Exploration dient dabei nur zur Steuerung der Navigation, zum Erstellen der Karte wird gmapping verwendet.

\subsection{Lokalisierung auf bekannten Karten}
Sofern der Turtlebot in einem bereits bekannten Gebiet fahren soll, kann gmapping nicht weiter eingesetzt werden. Zur reinen Lokalisierung auf geladenen Karten ist AMCL geeignet, welches eine adaptive Monte Carlo Lokalisierung auf Basis eines Partikelfilters einsetzt. Bei turtlebot_navigation liegt ebenfalls eine funktionierende Konfigurationsdatei bei, welche in diesem Projekt ungeändert eingesetzt wird.



