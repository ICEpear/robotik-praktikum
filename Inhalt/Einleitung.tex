\chapter{Einleitung}

Diese Dokumentation beschreibt die Arbeitsergebnisse des Robotik Praktikums welches an der Universität Oldenburg im Rahmen des Sommersemesters 2012 durchgeführt wurde. Während des Praktikums wurde ein Turtlebot (http://www.willowgarage.com/turtlebot) verwendet und das darauf laufende ROS System (http://www.ros.org) entsprechend der Aufgabenstellung des Praktikums angepasst.

\section{Aufgabenstellung}

Die Schwerpunkte des Praktikums lagen bei der autonomen Erstellung von Karten und dem Navigieren in so zuvor kartierten Räumen. Dabei sollten neben der bereits auf dem Turtlebot vorhandene Sensorik, welche vor allem Odometrie Sensoren und eine Kinect umfasst, auch ein Laserscanner zur Anwendung kommen. Dieser war dafür in das vorhandene System zu integrieren. Wobei auch über eine Rauschreduzierung des Laserscanners nachgedacht werden sollte. Zur Steuerung der entwickelten Komponenten war es das Ziel eine grafische Oberfläche zu implementieren. Eine weitere Aufgabe des Praktikums war das erstellen von Videos welche die Arbeitsergebnisse dokumentieren.